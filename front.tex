%!TEX TS-program = xelatex
%!TEX encoding = UTF-8 Unicode
% 參考資料
% https://www.foolegg.com/how-to-setting-up-a-xelatex-environment-on-osx/
% http://www.hitripod.com/blog/2011/04/xetex-chinese-font-cjk-latex/
\documentclass[9pt,a4paper]{book}
\usepackage{geometry} % 設定邊界
\geometry{
  top=0.7in,
  inner=0.4in,
  outer=0.4in,
  bottom=0.7in,
  headheight=3ex,
  headsep=2ex
}
\usepackage{multicol} % 多欄
\setlength\columnsep{1cm}
\usepackage{fontspec} % 允許設定字體
\usepackage{xeCJK} % 分開設置中英文字型
\setCJKmainfont[AutoFakeBold=3,AutoFakeSlant=.4]{DFKai-SB} % 標楷體
\setmainfont{PMingLiU}
\setCJKfamilyfont{songtitc}{Songti TC} % 宋體繁
\newcommand\fontst{\CJKfamily{songtitc}}
\setCJKfamilyfont{heiti}{WenQuanYi Micro Hei}
\newcommand\fontheiti{\CJKfamily{heiti}}
\usepackage{titlesec}
\titlespacing*{\subsection}{0pt}{1em}{1em}
% \setromanfont{Georgia} % 字型
% \setmonofont{Courier New}
\linespread{1.2}\selectfont % 行距
\XeTeXlinebreaklocale "zh" % 針對中文自動換行
\XeTeXlinebreakskip = 0pt plus 1pt % 字與字之間加入0pt至1pt的間距,確保左右對整齊
\parindent 0em % 段落縮進
\setlength{\parskip}{0pt} % 段落之間的距離
